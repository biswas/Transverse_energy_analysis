\chapter*{Abstract}\label{ch:abstract}

This thesis presents an analysis of the transverse energy resulting from the collisions of gold nuclei at the Relativistic Heavy Ion Collider in Brookhaven National Laboratory. The transverse momentum distributions available from the STAR detector corresponding to nine different centralities for eight different identified particles, $\pi^\pm$ [pions, anti-pions], $K^\pm$ [kaons, anti-kaons], $\Lambda^\pm$ [lambdas, anti-lambdas], $p$ [protons], and $\bar{p}$ [anti-protons], resulting from the collisions at five different center-of-mass energies per nucleon -- 7.7, 11.5, 19.6, 27, and 39 GeV -- are used in the calculations of the corresponding transverse energies. The results, when compared with the calorimetric transverse energy measurement from the PHENIX detector, show discrepancies of up to 2.83 $\sigma$ [standard deviations].
