\chapter{Method} \label{ch:method}
%\Conventional method using calorimeters
%\ first the math: definitions 1.3, 1.4 from analysis note
%\ then the description of the instruments used to get the relevant variables:
%\ 		primarily STAR (reference 12 from analysis note), include PHENIX and CMS later if needed

%\ transition to:

%\Spectra from the BES program
%\ Beam Energy Scan program
%\ particle identification
%\ transverse momentum spectrum (using tracking detectors????)
%\ errors associated with the spectra

%\Getting ET estimates from the spectra for individual particles
%\ equations relating dET/dy etc. to the pT integral of d2N/(dydpT)
%\ extrapolation of spectra to cover regions not covered by the experiment
%\ using ET estimates for individual particles to estimate total ET and the assumptions involved
%\ analysis note pg 7: "In this new method, ET is measured from charged hardon tracks...
%\ and the measured ET is SCALED UP to correct for neutral energy which is not observed...
%\ in tracking detectors."
%\ essentially, what really needs to be made clear is how the "scaling up" is done, because...
%\ even conventionally, STAR "uses information from the tracking detectors to measure ET from...
%\ charged hadrons and the electromagnetic calorimeter to measure ET from electrons, photons...
%\ and neutral hadrons which dominantly decay to electromagnetic particles." (hybrid method)

%\ centrality determination
In theory, $E_{T}$ from a collision can be defined as the sum of the transverse masses, $m_{T}$, of all the particles produced in the collision, i.e.,
\begin{equation}\label{eqn:ETDefTheory}
E_{T}\equiv\sum_{i}m_{T,i}
\end{equation}
with
\begin{equation}\label{eqn:mT}
m_{T}\equiv\sqrt{p_{T}^{2}+m^2}
\end{equation}
where $m$ is the rest mass of the particle and $p_{T}$ is its transverse momentum. Using this definition to calculate the $E_{T}$ requires perfect identification of all the particles. It has not been possible to do so in experiments, and so a more feasible, operational definition of $E_{T}$ is fabricated. A commonly accepted definition in case of the feasibility of calorimetric measurements is \cite{PhysRevC.89.044905, 1742-6596-458-1-012024}:
\begin{equation}\label{eqn:ETDefSum}
E_{T} = \sum_{i}E_{i}\sin{\theta_{i}},
\end{equation}
\begin{equation}\label{eqn:dETdEta}
\frac{dE_{T}}{d\eta}=\sin{\theta}\frac{dE}{d\eta},
\end{equation}

where the index $i$ runs over all the particles going into a fixed solid angle for each event, $\theta$ is the polar angle, i.e, the angle with respect to the beam axis, $\eta$ is the pseudorapidity defined as 
\begin{equation}\label{eqn:pseudorap}
\eta\equiv-\ln\tan{\frac{\theta}{2}},
\end{equation}
and $E_{i}$ is the energy deposited in the calorimeter by the $i^{th}$ particle. $E_{i}$ is considered to be, by convention \cite{PhysRevC.89.044905}???, the following
\begin{equation}\label{eqn:EiCaseByCase}
E_{i} = 
	\begin{cases}
	E_{i}^{tot}-m_{0} & \text{for baryons} \\
	E_{i}^{tot}+m_{0} & \text{for anti-baryons} \\	
	E_{i}^{tot} & \text{otherwise} \\
	\end{cases}
\end{equation}
%\ $E_{i}^{tot}$ - $m_{N}$ in case of baryons, $E_{i}^{tot}$ + $m_{N}$ in case of antibaryons, and the $E_{i}^{tot}$ in case of other particles, 
where $E_{i}^{tot}$ is the total energy of the $i^{th}$ particle defined canonically as
\begin{equation}\label{eqn:Etot}
E^{tot}\equiv\sqrt{p^{2}+m_{0}^2}
\end{equation}
and  $m_{0}$ is the particle's rest mass.
In order to account for the particles that are not identified by the calorimeters, corrections are made based on GEANT simulations of the collision and detection physics using models of the nucleus such as the Glauber model. (Doesn't that mean we use the model to make corrections in the analysis and the result of the analysis to judge the model????)

Corrections to account for unidentified particles are made for the tracking detectors also, but these turn out to be less than those made in case of the calorimeters. Hence, if the tracking detectors are able to give any information that can be used to estimate the total $E_{T}$, then we should be able to get a better estimate of the total ET than we would if we used the calorimeters. In fact, the tracking detectors in experiments such as the STAR (Solenoidal Tracker At RHIC) experiment and ALICE (A Large Ion Collider Experiment) at CERN include Time Projection Chambers (TPCs) and Time-of-Flight (TOF) detectors that can give us the pT spectra, yields and particle ratios of the identified charged hadrons \cite{Preghenella:2011vy, PhysRevC.96.044904}. The TPCs provide measurements of particle trajectories -- that can be used to determine the momenta for low-momentum particles -- and of their specific energy loss, 
\begin{equation}\label{eqn:specificEnLoss}
	\frac{dE}{dx} ,
\end{equation}
which can be used with the trajectories to make particle identifications using the Bethe-Bloch formula \cite{bethe1953passage}. TOF detectors, on the other hand, cover the high-momentum part of the measurements. In ALICE, the combination of the measurements of the TPC with those of the Inner Tracking System (ITS) effectively adds the tracking length, thereby improving the resolution of the measured pT spectrum. Details about the particle identification and momentum determination capabilities of the detectors in ALICE can be found in \cite{1748-0221-3-08-S08002}.

In the STAR experiment, the TPC is the primary tracking detector. It is 4.2 m long and it cylindrically enshrouds the beam pipe from its outside, with an inner diameter of 1 m and an outer diameter of 4 m \cite{phdthesisnattrass}. !!!!!!!!! more details about the TPC, then its limitation in high momentum resolution, then transition to TOF and some of its details !!!!!!!!!
%\Its drift volume is full of P10 gas (10% methane and 90% argon), the electrons from the molecules of which are knocked off by a charged particle travelling through the medium.

%\Spectra from the BES program (PhysRevC.96.044904 : https://arxiv.org/pdf/1701.07065.pdf)
%\ Beam Energy Scan program
%\ particle identification
%\ transverse momentum spectrum (using tracking detectors????)
%\ errors associated with the spectra





%\ NEXT: WRITE ABOUT THE ALTERNATIVE WAY, THAT IS, USING SPECTRA, AND WHAT DETECTOR THIS METHOD USES INSTEAD


%\"The main detectors used to obtain the results on pT spectra, yields and particle ratios for charged hadrons are the Time Projection Chamber (TPC) [41] and Time-Of Flight detectors (TOF) [42]" https://arxiv.org/pdf/1701.07065.pdf
%\ " The TPC data is used to determine particle trajectories, thereby their momenta, and particle types through ionization energy loss (dE/dx)."
%\ "For higher momentum, we use time-of-flight information to identify particles. The TOF particle identification for this analysis is used above pT = 0.4 GeV/c."

%\Historically, measurement of $E_{T}$ would be one of the first things done after heavy-ion collisions. Electromagnetic calorimeters (EMCals) would be used to perform the measurement of $E_{T}$. 
