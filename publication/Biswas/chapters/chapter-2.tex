\chapter{Theoretical Background} \label{ch:background}

% \section{Quantum Chromodynamics}

% \section{Phase Transitions}

% \section{Quark-Gluon Plasma}

\section{Quantum Chromodynamics}\label{section:QCD}
%%%%%%%%%%%%%%%%%%%%%%%%%%%%%%%%%%%%%%%%%%%%%%%%%%%%%%%%%%%%%%%%%%%%%%%%%%%%%%%
The strong force is one of the four fundamental interactions in physics. At large scales, it is also known as the residual strong force, and it is responsible for binding the nucleons together to give the nucleus its structure. At smaller scales, it is called the fundamental nuclear force, and it binds the fundamental units of subnuclear matter, the quarks, together to form the nucleons. The force carriers of the interaction are the mesons at the former scale and the gluons at the latter. %The scales of the different interactions and their relative strengths are summarized in table \ref{table:forces}.......... 
The electrodynamic interaction between charged particles such as protons and electrons is described by quantum electrodynamics (QED) as mediated by photons; the strong interaction, albeit more complicated, is explained under the framework of quantum chromodynamics (QCD) \cite{KAPUSTA1979461, Shuryak1988}. The quarks and gluons of QCD are collectively known as partons. Gluons are the gauge bosons of the Yang-Mills theory.

The Yang-Mills theory is a non-Abelian gauge theory. It has a Lagrangian with several degrees of freedom, some of which are redundant and need to be gauged. This is done by a mathematical treatment as prescribed under a gauge theory \cite{aitchison2003gauge}. The gauge theory associated with the Yang-Mills theory is based on the SU(N) group. It is non-Abelian as represented by the non-commutative transformations. QCD is a gauge theory that describes the application of the SU(3) symmetry transformations on color charges, namely red, blue, and green. The electroweak theory, which describes the electromagnetic as well as nuclear weak interactions, can be formalized under the gauge group SU(2)$\times$U(1). Together, they form the SU(3)$\times$SU(2)$\times$(U1) gauge theory called the standard model.
%Along with electric charge, mass and spin, quarks have the intrinsic property of color charge. 

One of the ways QCD is different from QED is the confinement of partons. In QED, the fundamental particles are bound together by the Coulomb potential, which diminishes with distance between the charge-carrying particles, as demonstrated by the relation \ref{eqn:QED-potential}:
\begin{equation}\label{eqn:QED-potential}
V_{C}\propto\frac{1}{r} 
\end{equation}
where $V_{C}$ is the Coulomb potential, and $r$ is the spatial separation between the particles. This means that bound QED particles can be isolated by increasing their spatial separation. The QCD potential, on the other hand, has an extra linear term in it%:
%(pg 7 https://www2.ph.ed.ac.uk/~muheim/teaching/np3/lect-qcd.pdf)
%(pg 68 https://arxiv.org/pdf/hep-ph/0001312.pdf)
, which means that the potential increases linearly with distance at large distances, and so an infinite amount of energy is required to separate quarks \cite{Bali:2000gf}. Hence, we never observe isolated quarks and they are said to be confined, not just bound, to form composite structures called hadrons \cite{0954-3899-32-3-R01}. A quark and an anti-quark forms a meson and three quarks forms a baryon. In addition to having a color charge, a quark also carries a flavor. There are six different quarks based on the flavors they carry: up, down, top, bottom, beauty, and strange.

%QCD is a gauge theory. Its Lagrangian remains unchanged under certain transformations. The Lagrangian has a number of degrees of freedom, some of which are redundant and need to be gauged, meaning regulated by a particular mathematical treatment. That mathematical treatment, in which the transformations of the gauge group are non-commutative, is described by a non-Abelian gauge theory. The Yang-Mills theory is an example of it. In this theory, the gauge boson is the gluon. 
%These confined, bound states of quarks and gluons is color-neutral.
%, and it is ergonomically more favorable to create a quark-antiquark pair than to produce unbound quarks
%, and $\alpha\_{QED}$ is the coupling constant.

\section{Phase Transitions}
%%%%%%%%%%%%%%%%%%%%%%%%%%%%%%%%%%%%%%%%%%%%%%%%%%%%%%%%%%%%%%%%%%%%%%%%%%%%%%%%%%%%%%%%%%%%%%
In everyday life, we observe matter existing in four distinct phases: solid, liquid, gas, and plasma. Changes in physical conditions can lead to a transition from one of these phases to another, exemplified by the commonly observed conversion of ice to water. Distinctions among the various phases can be represented in a chart called the phase diagram.

The phase diagram consists of thermodynamic observables such as temperature and density on its axes. Curves in the phase diagram represent boundries of physical conditions separating one phase from another: crossing a boundary represents an abrupt transition from one phase to another. This abruptness is mathematically characterized by the discontinuity in the change of the derivative of the free energy -- a thermodynamic variable -- with respect to the physical quantities in the axes. Such an abrupt transition is called a first-order phase transition. Along the boundary represented by the curve, there can be a point beyond which the phase transition is continuous instead of being abrupt, and the distinction between two phases is not clear. This point is called a critical point, and the phase transition that takes place beyond this point is called a crossover.% .............. Christine believes this is only for a first order phase transition................ 

One of the main focuses of current experimental and theoretical nuclear physics research is the study of the phase diagram of strongly interacting matter at a range of temperatures and baryon chemical potentials. In experiments involving the collisions of heavy ions at high and low energies, different regions of the phase diagram can be probed by varying the collision energy \cite{PhysRevC.93.024901}. For instance, the high-baryon chemical potential regime corresponds to lower beam energies and higher temperatures correspond to higher beam energies. The results of these experiments and model calculations can be used to study the possibilities and signatures of transitions in the QCD phase diagram.

A schematic representing the QCD phase diagram as a function of the temperature (T) and quark chemical potential ($\mu$) is shown in Fig. \ref{fig:PhaseDiagram} \cite{1742-6596-761-1-012066}. A crossover is predicted at low baryon chemical potentials (close to baryon-antibaryon symmetry) and high temperatures reminiscent of the early universe. Methods to study this region of the phase space will be explored in this thesis. At low temperatures and high net baryon densities, loose predictions have been made regarding the existence of exotic phases of high density matter, and programs, such as the Compressed Baryonic Matter experiment at the Facility for Antiproton and Ion Research in Germany, are being designed to study this region of the phase diagram \cite{HEUSER2013941c}.
% but within reach at modern facilities, specifically the Relativistic Heavy Ion Collider (RHIC) at the Brookhaven National Laboratory and the Large Hadron Collider (LHC) at CERN
%Details about these facilities are given in \ref{section:RHI-collisions}.
\begin{figure}[tb]
  \centering
  \includegraphics[width=4.5in]{figures/1742-6596-761-1-012066.png}\\
  \caption{Schematic of the QCD phase diagram \cite{1742-6596-761-1-012066}.}\label{fig:PhaseDiagram}
\end{figure}


\section{Quark-Gluon Plasma}
%%%%%%%%%%%%%%%%%%%%%%%%%%%%%%%%%%%%%%%%%%%%%%%%%%%%%%%%%%%%%%%%%%%%%%%%%%%%%%%%%%%%%%%%%%%%%%
The confinement of quarks into the hadronic phase of QCD matter, as described in section \ref{section:QCD}, has its limitations. At very high densities, when the wave function of a single hadron overlaps with the spatial regions covered by multiple such hadrons, it is impossible to classify which pair or triplet of quarks belongs to which meson or baryon. As long as a particular quark is close enough to the other quarks in the volume, it is deconfined in such a way that it can freely move anywhere in the volume \cite{0954-3899-32-3-R01}. QCD predicts such phase transition, at energy densities above 0.2-1 GeV/fm$^{3}$ \cite{Adam:2139456} and around a critical temperature of about 160 MeV \cite{FLORIS2014103}, of strongly interacting matter to a phase with quarks and gluons in thermal and chemical equilibrium representing the relevant degrees of freedom. This deconfined state of quarks and gluons is termed the quark-gluon plasma (QGP) in analogy to the quantum electrodynamical plasma phase of matter. The QGP has been found to behave like an almost perfect fluid \cite{PhysRevLett.109.152303}.
%\cite{FLORIS2014103 better reference than 2013arXiv1304.1452M}
%The deconfinement is what the weakening of the strong interaction due to the polarization of the QCD vacuum is expected to lead to at high energies. The expectation of this phase transition also makes sense in terms of the chiral symmetry of the QCD Lagrangian, which is spontaneously broken at low temperatures, but restored at high temperatures, providing a sufficient condition for the deconfinement.
% Existence of QGP in the early universe
% Production of QGP in the lab
