\chapter{Conclusion and Future Work} \label{ch:conclusion}
I calculated the transverse energy from Au+Au collisions at $\sqrt{s_{NN}}$ = 7.7, 11.5, 19.6, 27, and 39 GeV for nine different centralities using the transverse momentum spectra available for identified pions, kaons and protons (and their anti-particles) and also the preliminary transverse momentum spectra available for lambdas and anti-lambdas. The Boltzmann-Gibbs Blast Wave model was used to extrapolate the spectra in the $p_{T}$ regions where data were not available. I calculated the total $E_{T}$ from these spectra by assuming that the contributions from heavier partiles are negligible and that the different isospin states of a particle carry roughly the same amount of $E_{T}$ in the aftermath of a relativistic heavy ion collision. My results show that the shapes of the distributions of $E_{T}$ found from the analyis using the tracking detectors in the STAR experiment are similar to those found from the results from the electromagnetic calorimeters in the PHENIX experiment. However, the values I calculate for $(dE_{T}/d\eta)/0.5N_{part}$ differ from the corresponding PHENIX calculations by 1.67 to 2.83 standard deviations at different collision energies. Some investigation needs to be done into what the possible causes for this discrepancy might be before publication.% This discrepancy needs to be explained in the future before the codes I developed for this analysis is used for other similar analyses.

There are other aspects of this analysis that can be improved in the future. For instance, a maximum likelihood fit method can be adopted to compare the results with those using the chi-squared fits in the extrapolation of the spectra. Apart from the transverse energy, the calculation of the initial energy density, $\epsilon$, as given by the Bjorken formula in eq. \ref{eqn:bjorken}, is possible with some more effort. It requires the estimate of other physical quantities. Adare et al.\cite{PhysRevC.93.024901} use the Glauber model to determine $A_{T}$, the area of the intersection of the two nuclei in the transverse plane. Since the results in this thesis are cross-checked with those in \cite{PhysRevC.93.024901}, it would be reasonable to use the same model in the future work pertaining to this thesis.% $\tau_{0}$, the proper time at the moment of QGP equilibration, also depends on the model of the collision. However, the product of $\epsilon$ and $\tau_{0}$ is often used instead of just $\epsilon$ to study how the energy density scales with the collision energy and the number of participants.

Finally, the codes in the repository can be used to analyze the data pertaining to other different collision systems. Since there is more data available on collisions of asymmetric systems such as d+Au, we can expect it to be a test to tell if the assumptions used in this analysis are reasonable for such systems. More importantly, this technique can be used to analyze higher energy RHIC data to find out if STAR spectral $E_{T}$ agrees better with PHENIX calorimetric $E_{T}$ when more energy is available at the time of collision.

