\chapter{Introduction} \label{ch:introduction}

The Large Hadron Collider (LHC) at CERN and the Relativistic Heavy Ion Collider (RHIC) at the Brookhaven National Laboratory have the ability to collide heavy nuclei, such as those of gold and uranium, at nearly the speed of light, reaching temperatures of trillions of degrees Celcius. These laboratories have provided evidence of the formation of an exotic state of matter, called the quark-gluon plasma (QGP). It only exists for a brief amount of time after such collisions and instantly freezes out into a plethora of new particles, which carry the signatures we can use to deduct QGP properties. It reportedly behaves like an almost perfect quantum fluid with no resistance and exhibits other interesting properties.

One of the methods to probe the properties of this matter is by analyzing the conversion of the beam-direction energy at the time of collision into transverse energy after the collision. This analysis is generally done by using data from the calorimeters placed around the collision site. In this thesis, I use the data collected by the tracking detectors, instead of the conventional calorimeters, to perform the transverse energy analysis.

The organization of the thesis is as follows. In Chapter 2, I attempt to summarize the physical concepts pertaining to nuclear matter, heavy-ion collisions, and the production and detection of QGP. Chapter 3 consists of the formalism of the measurement of transverse energy using calorimeters as well as tracking detectors. It also gives an example of what has been done using calorimeters. Chapter 4 describes the data used to perform the analysis in this thesis and notes down the details of the analysis. In Chapter 5, I present the results and compare them to the ones in literature obtained using a different method. Chapter 6 concludes the thesis by summarizing it and shedding light on some of its implications.

Following parts need to be put into other relevant sections ********************

The existence and properties of the QGP in the aftermath of high-energy heavy-ion collisions can be probed using different techniques relevant to several theoretical characteristics of the phase. For instance, the interacting nuclei  carry no net strangeness before colliding, and so a post-collision observation of strange and multi-strange particles can be a signal for an antecedent existence of deconfined quarks and gluons \cite{1742-6596-455-1-012005}. This signal, when complemented with an observation of the suppression???????or enhancement of strange particles production, provides a strong hint of the formation of QGP. This can be further complemented with the estimate of the energy density and the temperature attained after the collision.

Analyses of experimental results have thus far provided signatures of the formation of matter with partonic degrees of freedom at the early stages of the collisions. Such signatures include suppression of high monentum hadrons, known as jet quenching, because the QGP is nearly opaque to colored probes, and large azimuthal anisotropies, indicating that the medium is a liquid of quarks and gluons \cite{PhysRevC.96.044904}?????. Experiments also reveal the initial energy density of this matter to be about two orders of magnitude larger than that of low energy nuclear matter -- comfortably more than the deconfinement phase transition critical density predicted by lattice QCD \cite{2005PrPNP..54..443J}.

The state of the colliding nuclei before the collision at LHC and top RHIC energies has indications of being a Color Glass Condensate -- strongly interacting, weakly coupled highly coherent gluonic matter \cite{1742-6596-458-1-012024}. The characteristics of the initial states of these nuclei affect the partonic distributions within the nuclei and ultimately the products of the collision. The collision products are also affected by variables such as the initial energy and entropy densities of the partonic matter \cite{2005PrPNP..54..443J}.

Different observables can be used to study different aspects of heavy ion collisions. The charged particle multiplicity, $\langle N_{ch} \rangle$, is a global variable that relates to the entropy production during the collision (analysis note). The transverse energy, $E_{T}$, a global variable related to $\langle N_{ch} \rangle$, provides information about the conversion of the initial beam-direction kinetic energy into energy flowing in the transverse direction after the collision. Together, the studies of the fluctuation of the $\langle N_{ch} \rangle$ and the $E_{T}$ pseudorapidity [footnote] density with respect to the beam energy and the collision centrality [footnote] help probe the characteristics of the initial conditions at the time of the collision. One can study, for instance, the distinctions between models based on quark participants against those based on nucleon participants [analysis note]. These quantities can also lead to the rough estimate of the initial energy density through the use of the Bjorken formula \cite{2012ARNPS..62..361M}:
%\ref{eqn:Bjorken}
\begin{equation}\label{eqn:Bjorken}
\epsilon \geq \frac{\frac{dE_{T}}{d\eta}}{\tau_{0}\pi R^{2}} = \frac{3}{2}\langle \frac{E_{T}}{N} \rangle \frac{\frac{dN_{ch}}{d\eta}}{\tau_{0}\pi R^{2}}
\end{equation}

The transverse energy and the charged particle pseudorapidity densities have conventionally been calculated by using the transverse energy measurements obtained from calorimeters. This thesis details the use of particle spectra, reported as $\frac{d^{2}N}{dydp_{T}}$, from Au+Au collisions at RHIC to calculate the same global variables and serve as a method to cross check the ones involving calorimeters.

The organization of the thesis is as follows. Chapter II contains brief descriptions of different conventional methods used to estimate $E_{T}$ as well as an elaboration of the method specific to this thesis.
